\documentclass[dvipdfmx,a4paper,11pt]{article}
\usepackage[utf8]{inputenc}
%\usepackage[dvipdfmx]{hyperref} %リンクを有効にする
\usepackage{url} %同上
\usepackage{amsmath,amssymb} %もちろん
\usepackage{amsfonts,amsthm,mathtools} %もちろん
\usepackage{braket,physics} %あると便利なやつ
\usepackage{bm} %ラプラシアンで使った
\usepackage[top=30truemm,bottom=20truemm,left=25truemm,right=25truemm]{geometry} %余白設定
\usepackage{latexsym} %ごくたまに必要になる
\renewcommand{\kanjifamilydefault}{\gtdefault}
\usepackage{otf} %宗教上の理由でmin10が嫌いなので


\usepackage[all]{xy}
\usepackage{amsthm,amsmath,amssymb,comment}
\usepackage{amsmath}    % \UTF{00E6}\UTF{0095}°\UTF{00E5}\UTF{00AD}\UTF{00A6}\UTF{00E7}\UTF{0094}¨
\usepackage{amssymb}  
\usepackage{color}
\usepackage{amscd}
\usepackage{amsthm}  
\usepackage{wrapfig}
\usepackage{comment}	
\usepackage{graphicx}
\usepackage{setspace}
\usepackage{pxrubrica}
\usepackage{enumitem}
\usepackage{mathrsfs} 

\setstretch{1.2}


\newcommand{\R}{\mathbb{R}}
\newcommand{\Z}{\mathbb{Z}}
\newcommand{\Q}{\mathbb{Q}} 
\newcommand{\N}{\mathbb{N}}
\newcommand{\C}{\mathbb{C}} 
\newcommand{\D}{\mathbb{D}} 
%\newcommand{\H}{\mathbb{H}} 
\newcommand{\Sin}{\text{Sin}^{-1}} 
\newcommand{\Cos}{\text{Cos}^{-1}} 
\newcommand{\Tan}{\text{Tan}^{-1}} 
\newcommand{\invsin}{\text{Sin}^{-1}} 
\newcommand{\invcos}{\text{Cos}^{-1}} 
\newcommand{\invtan}{\text{Tan}^{-1}} 
\newcommand{\Area}{\text{Area}}
\newcommand{\vol}{\text{Vol}}
\newcommand{\maru}[1]{\raise0.2ex\hbox{\textcircled{\tiny{#1}}}}
\newcommand{\sgn}{{\rm sgn}}
%\newcommand{\rank}{{\rm rank}}



   %当然のようにやる.
\allowdisplaybreaks[4]
   %もちろん.
%\title{第1回. 多変数の連続写像 (岩井雅崇, 2020/10/06)}
%\author{岩井雅崇}
%\date{2020/10/06}
%ここまで今回の記事関係ない
\usepackage{tcolorbox}
\tcbuselibrary{breakable, skins, theorems}

\theoremstyle{definition}
\newtheorem{thm}{定理}
\newtheorem{lem}[thm]{補題}
\newtheorem{prop}[thm]{命題}
\newtheorem{cor}[thm]{系}
\newtheorem{claim}[thm]{主張}
\newtheorem{dfn}[thm]{定義}
\newtheorem{rem}[thm]{注意}
\newtheorem{exa}[thm]{例}
\newtheorem{conj}[thm]{予想}
\newtheorem{prob}[thm]{問題}
\newtheorem{rema}[thm]{補足}

\DeclareMathOperator{\Ric}{Ric}
\DeclareMathOperator{\Vol}{Vol}
 \newcommand{\pdrv}[2]{\frac{\partial #1}{\partial #2}}
 \newcommand{\drv}[2]{\frac{d #1}{d#2}}
  \newcommand{\ppdrv}[3]{\frac{\partial #1}{\partial #2 \partial #3}}


%ここから本文.
\begin{document}
%\maketitle

\begin{center}
{\Large 4 有理型関数・バーゼル問題}
\end{center}

\begin{flushright}
 岩井雅崇 2023/04/11
\end{flushright}
以下断りがなければ, $\Omega$は$\C$の領域(連結開集合)とし, $\D=\{z \in \C |  |z| <1\}$とする. 

\vspace{12pt}
\hspace{-24pt}\underline{ローラン展開・極に関する問題}

\begin{enumerate}[label=\textbf{問}4.\arabic*]

\item $^{\bullet}$次の問いに答えよ. %\footnote{問題の都合上$f$や$a$が何かは言えないが,.}
   \begin{enumerate}
 \setlength{\parskip}{0cm} 
  \setlength{\itemsep}{0cm} 
  \item 「$a$が$f$の除去可能特異点である」, 「$a$が$f$の極である」, 「$a$が$f$の真性特異点である」ことの定義をそれぞれ述べよ.
  \item 「$f$が$\Omega$の有理型関数である」, 「$f$が$\Omega$の有理関数である」ことの定義をそれぞれ述べよ. また二つの定義の違いは何か?\footnote{有理型関数だが有理関数でないものの例をあげても良い. }
    \end{enumerate}
    
 
\item $^{\bullet}$ 「$\frac{e^z}{z^3}$の$z=0$での留数」と「$\frac{z^4 + 3}{z^4 -1}$の$z=i$での留数」をそれぞれ求めよ.
%\item $^{\bullet}$ $\frac{z^4 + 3}{z^4 -1}$の$z=i$での留数を求めよ.
\item $^{\bullet}$ $e^{z + \frac{1}{z}}$の$z=0$でのローラン展開を求めよ.

\item $^{\bullet}$ (Chat GPTによる問題) 以下の関数の留数や特異点を求めよ.\footnote{Chat GPTに「複素関数論続論演義の問題の案を教えて」と言ったらこの問題が返ってきた. その他いろいろと試してみるとChat GPTは私の演習問題を5-7割くらい解けるようである. }

(a) $f(z) = \frac{1}{(z-1)^2(z+2)}$,  
(b) $f(z) = \frac{\sin z}{z^3(z-\pi)^2}$


\item $\frac{1}{(z-1)(z-2)}$の円環領域$\{ z\in \C | 1 < |z| < 2\}$におけるローラン展開を求めよ.

\item  $^{*}$ $\frac{1}{(\sin z)^2}$の$z=0$におけるローラン展開を$z^4$の係数まで求めよ.

 \item $f$を極が有限個である$\C$上の有理型関数で, $\lim_{|z| \rightarrow \infty} = \alpha $となる$\alpha \in \C$が存在すると仮定する. このとき$f(z)$は有理関数であることを示せ.
   また「$\lim_{|z| \rightarrow \infty} = \alpha $」という仮定を外したとき, この主張は成り立つか?
       
  \item \label{henkaku} (偏角の原理) $f$を円盤$D$上の正則関数とし, $\partial D$上に$f$の零点はないものとする. 
  $f$の$D$での零点の個数を$N$とする (ただし, $m$位の零点は$m$この零点と重複して考える.)
  このとき
  $$
  \frac{1}{2 \pi i} \int_{\partial D} \frac{f' (z)}{f(z)} dz = N
  $$
  が成り立つことを示せ. また$f$が有理型関数の場合はどうなるか?
  \item \ref{henkaku}を用いて代数学の基本定理を示せ.
  
\vspace{12pt}
\hspace{-36pt}\underline{正則関数の拡張に関する問題}



 \item $^{\bullet}$  $\D^{*}=\{z \in \C | 0 < |z| <1\}$とし, $f$を$\D^{*}$上の正則関数とする. 
 $f$が$\D^{*}$上で有界ならば, $f$は$\D$上の正則関数に拡張できることを示せ. 
 
  \item $\D^{*}=\{z \in \C | 0 < |z| <1\}$とし, 
  $f$を$\D^{*}$上の正則関数とする. 
 $z \in \D^{*}$によらない定数$M>0$があって, $\D^{*}$上で$|f(z)| \le M \log |z|$
 を満たすとき, $f$は$\D$上の正則関数に拡張できることを示せ. \footnote{もっと強く$f$が$L^2$関数なら拡張できる.(演習でこれを示しても良い).}
 
 
 \item $(-1,1) \setminus \{ 0\}$上の有界な$C^{\infty}$級関数で$(-1,1)$に$C^{\infty}$級拡張できないものを構成せよ. 
 
  \item $\C \setminus \{ 0\}$上の正則関数$f(z)=\frac{1 - \cos z}{z^2}$は$f(0)$をうまく定めれば$\C$上の正則関数に拡張できることを示せ. 
  
 \item $f(z)$を$\C$上の正則関数とする. 任意の$z \in \C$について$|f(z)| \le |\sin z|$が成り立つならば, ある$a \in \C$が存在して$f(z) = a \sin z$となることを示せ. 
 


      
 \vspace{12pt}
\hspace{-36pt}\underline{第5回授業に関する問題}     
  \item (授業の内容)$\sum_{n= - \infty}^{\infty} \frac{1}{(z - n \pi)^2}$は領域$\{ z \in \C | |{\rm Im}(z)| >1, 0 < {\rm Re}(z) < 1\}$上において一様収束することを示せ.

 \item  (授業の内容)次の問いに答えよ.
  \begin{enumerate}
\setlength{\parskip}{0cm} 
  \setlength{\itemsep}{0cm} 
\item $\sin z =0$となる$z \in \C$を全て求めよ. 
\item $\sin z =2$となる$z \in \C$を全て求めよ. 
   \end{enumerate}
   
 \item  (授業の内容)$|{ \rm Im}(z)| \rightarrow \infty$ならば$|\sin z| \rightarrow \infty$となることを示せ.
 
 
\item $\sum_{n=1}^{\infty}\frac{1}{n^4}$と$\sum_{n=1}^{\infty}\frac{1}{n^6}$をそれぞれ求めよ. 

   
 
\vspace{12pt}
\hspace{-36pt}\underline{院試の問題}  
  \item $^{*}$ $a>0$とし
  $$
  f(z) = \frac{ e^{iz}}{z(z^2 + a^2)}
  $$
  と定める. 次の問いにこたえよ.
   \begin{enumerate}
 \setlength{\parskip}{0cm} 
  \setlength{\itemsep}{0cm} 
  \item $z=0$における$f(z)$のローラン展開の主要部を求めよ.
  \item $r>0$とし$\C$上の積分路を$C_{r} : z = r e^{i \theta} (0 \le \theta \le \pi)$とおく. 
  $I(r) := \int_{C_r} f(z )dz$とおくとき, $\lim_{r \rightarrow 0} I(r)$と$\lim_{r \rightarrow \infty} I(r)$の値をそれぞれ求めよ.
  \item $0 < r < a < R$なる実数に対し
  $$
  D_{r,R}= \{ z \in \C | r < |z| < R, {\rm Im}(z) >0 \}
  $$
  とおく. $\int_{\partial D_{r,R}} f(z) dz$の値を求めよ.
  \item 広義積分$\int_{0}^{\infty}\frac{ \sin x}{x(x^2 + a^2)} dx$は収束することを示し, その値を求めよ.
      \end{enumerate}
     
\item $^{*}$大阪大学の数学科の院試の問題で複素解析に関係あるものを解け. ただし解答前に教官(岩井)に問題を見せること. 

    \end{enumerate}      
 

 
 \vspace{11pt}\begin{wrapfigure}{r}[0pt]{0.2\textwidth}  \centering\includegraphics[height=25mm, width=25mm]{complex.png}\end{wrapfigure}

演習の問題は授業ページ(\url{https://masataka123.github.io/2023_summer_complex/})にもあります. 右下のQRコードからを読み込んでも構いません.


  
  
 \end{document}
