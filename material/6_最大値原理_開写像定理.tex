\documentclass[dvipdfmx,a4paper,11pt]{article}
\usepackage[utf8]{inputenc}
%\usepackage[dvipdfmx]{hyperref} %リンクを有効にする
\usepackage{url} %同上
\usepackage{amsmath,amssymb} %もちろん
\usepackage{amsfonts,amsthm,mathtools} %もちろん
\usepackage{braket,physics} %あると便利なやつ
\usepackage{bm} %ラプラシアンで使った
\usepackage[top=30truemm,bottom=20truemm,left=25truemm,right=25truemm]{geometry} %余白設定
\usepackage{latexsym} %ごくたまに必要になる
\renewcommand{\kanjifamilydefault}{\gtdefault}
\usepackage{otf} %宗教上の理由でmin10が嫌いなので


\usepackage[all]{xy}
\usepackage{amsthm,amsmath,amssymb,comment}
\usepackage{amsmath}    % \UTF{00E6}\UTF{0095}°\UTF{00E5}\UTF{00AD}\UTF{00A6}\UTF{00E7}\UTF{0094}¨
\usepackage{amssymb}  
\usepackage{color}
\usepackage{amscd}
\usepackage{amsthm}  
\usepackage{wrapfig}
\usepackage{comment}	
\usepackage{graphicx}
\usepackage{setspace}
\usepackage{pxrubrica}
\usepackage{enumitem}
\usepackage{mathrsfs} 

\setstretch{1.2}


\newcommand{\R}{\mathbb{R}}
\newcommand{\Z}{\mathbb{Z}}
\newcommand{\Q}{\mathbb{Q}} 
\newcommand{\N}{\mathbb{N}}
\newcommand{\C}{\mathbb{C}} 
\newcommand{\D}{\mathbb{D}} 
%\newcommand{\H}{\mathbb{H}} 
\newcommand{\Sin}{\text{Sin}^{-1}} 
\newcommand{\Cos}{\text{Cos}^{-1}} 
\newcommand{\Tan}{\text{Tan}^{-1}} 
\newcommand{\invsin}{\text{Sin}^{-1}} 
\newcommand{\invcos}{\text{Cos}^{-1}} 
\newcommand{\invtan}{\text{Tan}^{-1}} 
\newcommand{\Area}{\text{Area}}
\newcommand{\vol}{\text{Vol}}
\newcommand{\maru}[1]{\raise0.2ex\hbox{\textcircled{\tiny{#1}}}}
\newcommand{\sgn}{{\rm sgn}}
%\newcommand{\rank}{{\rm rank}}



   %当然のようにやる.
\allowdisplaybreaks[4]
   %もちろん.
%\title{第1回. 多変数の連続写像 (岩井雅崇, 2020/10/06)}
%\author{岩井雅崇}
%\date{2020/10/06}
%ここまで今回の記事関係ない
\usepackage{tcolorbox}
\tcbuselibrary{breakable, skins, theorems}

\theoremstyle{definition}
\newtheorem{thm}{定理}
\newtheorem{lem}[thm]{補題}
\newtheorem{prop}[thm]{命題}
\newtheorem{cor}[thm]{系}
\newtheorem{claim}[thm]{主張}
\newtheorem{dfn}[thm]{定義}
\newtheorem{rem}[thm]{注意}
\newtheorem{exa}[thm]{例}
\newtheorem{conj}[thm]{予想}
\newtheorem{prob}[thm]{問題}
\newtheorem{rema}[thm]{補足}

\DeclareMathOperator{\Ric}{Ric}
\DeclareMathOperator{\Vol}{Vol}
 \newcommand{\pdrv}[2]{\frac{\partial #1}{\partial #2}}
 \newcommand{\drv}[2]{\frac{d #1}{d#2}}
  \newcommand{\ppdrv}[3]{\frac{\partial #1}{\partial #2 \partial #3}}


%ここから本文.
\begin{document}
%\maketitle

\begin{center}
{\Large 6 最大値原理・開写像定理}
\end{center}

\begin{flushright}
 岩井雅崇 2023/05/23
\end{flushright}
以下断りがなければ, $\Omega$は$\C$の領域(連結開集合)とし, $\D=\{z \in \C |  |z| <1\}$とする. 

%\vspace{12pt}\hspace{-24pt}\underline{ローラン展開・極に関する問題}

\begin{enumerate}[label=\textbf{問}6.\arabic*]

\item $^{\bullet}$ $f$を$|f| \le 1$となる$\D$上の正則関数とする. $f(\frac{1}{2})=1$ならば$f$は定数関数であることを示せ.

\item $^{\bullet}$ $f$を$\D$上の正則関数かつ$\bar{\D}$上で連続な関数とし, $f$は$\D$内で零点を持たないものとする. 
$|z|=1$上で$|f(z)|$が定数関数ならば, $f$は$\D$上で定数関数であることを示せ. (ヒント: $f, \frac{1}{f}$に最大値原理を用いよ. もしくは最大値原理と最小値原理を用いても良い. )
%$f$を$\D$上の正則関数かつ$\bar{\D}$上で連続な関数とする.$|z|=1$である任意の$z$について$|f(z)|=0$ならば, $f$は定数関数であることをしめせ.

\item $^{\bullet}$次の問いに答えよ.
  \begin{enumerate}
\setlength{\parskip}{0cm} 
  \setlength{\itemsep}{0cm} 
\item  $p : \R^{2} \rightarrow \R$, $p(x,y)=x$は開写像であるが閉写像ではないことを示せ. 
\item 開写像定理の主張を述べよ.
\item $q : \R^{2} \rightarrow \R^2$, $q(x,y)=(x,xy)$は開写像ではないことを示せ.  (つまり「開写像定理」において正則性は必要である.)
%定数でない$C^{\infty}$級関数で開写像ではないものの例をあげよ. (つまり「開写像定理」において正則性は必要である.)
    \end{enumerate}  
   

\item $^{\bullet}$次の問いに答えよ.
  \begin{enumerate}
\setlength{\parskip}{0cm} 
  \setlength{\itemsep}{0cm} 
  \item $C^{\infty}$級関数$f : \R \rightarrow \R$で単射であるが, ある$a \in \R$があって$f'(a)=0$となるものを一つ構成せよ.
   \item 「逆関数の正則性」の主張を述べよ.
\item $C^\infty$級関数$f : \R \rightarrow \R$で全単射であるが$f^{-1}$が$C^{\infty}$級でないものを一つ構成せよ. (つまり「逆関数の正則性」において正則性は必要である.)
   \end{enumerate}  
   
   
   \item $f$を$\Omega$上の定数でない正則関数とする. $a \in \Omega$があって$f' (a) \neq 0$であるならば, 
   $a$を含む半径$r > 0$の円板$D$で$f|D : D \rightarrow \C$が単射になるものが存在することを示せ.
    
   \item $f$を$\Omega$上の単射な正則関数とし, $\bar{D} \subset \Omega$となる円板$D$とする. 
   任意の$w \in f(D)$について次を示せ. 
   $$
   f^{-1}(w) = \frac{1}{2 \pi i} \int_{\partial D} \frac{zf'(z)}{f(z) - w} dz
   $$
   
   \item $f$を$\D$上の正則関数とする. 次の問いに答えよ.
     \begin{enumerate}
\setlength{\parskip}{0cm} 
  \setlength{\itemsep}{0cm} 
  \item 
  $
  \lim_{r \to 1} 
  \left(\sup_{ \theta \in [0 , 2 \pi]} |f(r e^{i \theta})| \right) =0
  $
 ならば$f \equiv 0$であることを示せ.
\item     $
  \lim_{r \to 1} 
  \left(\sup_{ \theta \in [0 , \frac{\pi}{2}] } |f(r e^{i \theta})| \right) =0
  $
 ならば$f \equiv 0$であることを示せ. (ヒント: $f$をうまく組み合わせて(a)を満たす関数を作れ.)
%$\theta \in [0, \frac{\pi}{2}]$に対して$|z| \rightarrow 1$とするとき$f$が0に一様収束するならば$f \equiv 0$であることを示せ.
        \end{enumerate}  
\item $w_1, \ldots, w_n$を$\C$上の単位円周上$S^1(=\partial \D)$の点とする. 
次の問いに答えよ.
  \begin{enumerate}
\setlength{\parskip}{0cm} 
  \setlength{\itemsep}{0cm} 
\item $\prod_{i=1}^{n} |z - w_i| \ge 1$となる$z \in S^1$が存在することを示せ. 
\item $\prod_{i=1}^{n} |z - w_i|=1$となる$z \in S^1$が存在することを示せ. 
     \end{enumerate}  
     
     
     \newpage
   \item 次の問いに答えよ.
     \begin{enumerate}
\setlength{\parskip}{0cm} 
  \setlength{\itemsep}{0cm} 
 \item (カゾラティ-ワイエルシュトラスの定理) $f$を$\D \setminus \{ 0\}$上の正則関数とする. $f$が$0$で真性特異点を持つならば, $f(\D \setminus \{ 0\})$は$\C$において稠密であることを示せ. \footnote{ヒント: もしそうでないなら, ある点$w \in \C $があって$g(z):=\frac{1}{f(z) - w}$が$\D \setminus \{ 0\}$上で有界となってしまう. これは$f$が$0$で真性特異点を持つことに反する(なぜか?)}
\item  $f : \C \rightarrow \C$が単射な正則関数ならば, ある$a,b \in\C$があって$f(z) =az +b$であることを示せ. (ヒント: $f(1/z)$を考えよ.)
       \end{enumerate}  
\item $^{*}$$\C^{2} \setminus \{ 0\}$について, 同値関係$\sim$を「
	$
	z \sim w \Leftrightarrow \text{0でない複素数$\alpha$が存在して$z = \alpha w$}
	$」
	と定義する. $ \C\mathbb{P}^{1}:= (\C^{2} \setminus \{ 0\})/\sim$と書き複素射影空間と呼ぶ. 以下$z = (z_{1}, z_{2})$を$\C\mathbb{P}^{1}$の元とみなしたものを$(z_{1}:z_{2})$と書き複素同次座標と呼ぶ.
$U_{1} = \{ (z_{1}:z_{2}) | z_{1}\neq 0\}, U_{2} = \{ (z_{1}:z_{2}) | z_{2}\neq 0\}$とおき, 
$$
\begin{array}{cccccccccc}
\varphi_{1}: &\C& \rightarrow & U_{1}& &\varphi_{2}: &\C& \rightarrow & U_{2}& \\
                      &z &\longmapsto &(1:z)&      &                      &w& \longmapsto &(w:1)&
\end{array}
$$	
と定める. 次の問いに答えよ.
  \begin{enumerate}
\setlength{\parskip}{0cm} 
  \setlength{\itemsep}{0cm} 
  \item $\varphi_1, \varphi_2$は全単射であることをしめせ.  また逆写像を各々構成せよ.\footnote{もっと強く同相であるが, 面倒なのでそこまで示さなくても良い.}
  \item $f : \C\mathbb{P}^{1} \rightarrow  \C$を連続写像とする. $f \circ \varphi_{1}$と$f \circ \varphi_{2}$が共に正則関数であるならば, $f$は定数関数であることを示せ. 
     \end{enumerate}  
     
 \item $^{*}$(開写像定理を用いた代数学の基本定理の証明) $f(z) = z^n + a_1 z^{n-1}+ \cdots + a_{n-1}z + a_n$とし, $n \ge 1, a_n \neq 0$とする.  以下の問いに答えよ.
   \begin{enumerate}
\setlength{\parskip}{0cm} 
  \setlength{\itemsep}{0cm} 
\item $F : \C^2 \rightarrow \C^2$を$F(z,w) = (z^n + a_1 z^{n-1}w+ \cdots +a_{n-1}zw^{n-1} + a_n w^n, w^n )$とすると, $F$は連続写像$\tilde{F} : \C\mathbb{P}^{1} \to \C\mathbb{P}^{1} $を誘導することを示せ. 
\item $f(\C)$は閉集合であることをしめせ.
\item $f(\C)=\C$を示せ. これより$f(\alpha)=0$なる$\alpha \in \C$は存在する.
      \end{enumerate}  

   \item (ルーシェの定理)
   $f,g$を$\Omega$上の正則関数, $D \subset \Omega$を円板, $C = \partial D$を円周とする.
   $C$上で$|f(z)| > |g(z)|$ならば, $f$と$f+g$は$D$内で(重複度込みで)同じ個数の零点を持つことを示せ.
   (ヒント: $t \in [0,1]$について$n_{t} : = \frac{1}{2 \pi i} \int_{C} \frac{f' + t g'}{f +t g} dz$とすると$n_t$が$[0,1]$上の整数値連続関数であることを示せ. 授業の開写像定理の証明も参考にせよ.)
   
 %  \item ルーシェの定理を用いて代数学の基本定理を示せ.
   
   \item $z^4 - 6z + 3=0$は$1 < |z| < 2$内に何個の解を持つか?
 
   
   \item $^{*}$ $|z| \le 3$の近傍で定義された正則関数$f(z)$で$|z|=3$上では零点を持たないものとする. 
   $f$が次の3条件を満たすとき, $f(z)$の$|z |<3$における零点を全て求めよ.
   $$
\frac{1}{2 \pi i} \int_{|z|=3} \frac{f'(z)}{f(z)} dz =2, \quad
\frac{1}{2 \pi i} \int_{|z|=3} \frac{zf'(z)}{f(z)} dz =2, \quad
\frac{1}{2 \pi i} \int_{|z|=3} \frac{z^2f'(z)}{f(z)} dz =-4 
   $$

 
     \end{enumerate}  
 
 \vspace{11pt}\begin{wrapfigure}{r}[0pt]{0.2\textwidth}  \centering\includegraphics[height=20mm, width=20mm]{complex.png}\end{wrapfigure}

演習の問題は授業ページ(\url{https://masataka123.github.io/2023_summer_complex/})にもあります. 右下のQRコードからを読み込んでも構いません.


  
  
 \end{document}
