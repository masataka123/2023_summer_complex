\documentclass[dvipdfmx,a4paper,11pt]{article}
\usepackage[utf8]{inputenc}
%\usepackage[dvipdfmx]{hyperref} %リンクを有効にする
\usepackage{url} %同上
\usepackage{amsmath,amssymb} %もちろん
\usepackage{amsfonts,amsthm,mathtools} %もちろん
\usepackage{braket,physics} %あると便利なやつ
\usepackage{bm} %ラプラシアンで使った
\usepackage[top=30truemm,bottom=30truemm,left=25truemm,right=25truemm]{geometry} %余白設定
\usepackage{latexsym} %ごくたまに必要になる
\renewcommand{\kanjifamilydefault}{\gtdefault}
\usepackage{otf} %宗教上の理由でmin10が嫌いなので


\usepackage[all]{xy}
\usepackage{amsthm,amsmath,amssymb,comment}
\usepackage{amsmath}    % \UTF{00E6}\UTF{0095}°\UTF{00E5}\UTF{00AD}\UTF{00A6}\UTF{00E7}\UTF{0094}¨
\usepackage{amssymb}  
\usepackage{color}
\usepackage{amscd}
\usepackage{amsthm}  
\usepackage{wrapfig}
\usepackage{comment}	
\usepackage{graphicx}
\usepackage{setspace}
\usepackage{pxrubrica}
\usepackage{enumitem}
\usepackage{mathrsfs} 
\usepackage{caption}

\setstretch{1.2}


\newcommand{\R}{\mathbb{R}}
\newcommand{\Z}{\mathbb{Z}}
\newcommand{\Q}{\mathbb{Q}} 
\newcommand{\N}{\mathbb{N}}
\newcommand{\C}{\mathbb{C}} 
\newcommand{\Sin}{\text{Sin}^{-1}} 
\newcommand{\Cos}{\text{Cos}^{-1}} 
\newcommand{\Tan}{\text{Tan}^{-1}} 
\newcommand{\invsin}{\text{Sin}^{-1}} 
\newcommand{\invcos}{\text{Cos}^{-1}} 
\newcommand{\invtan}{\text{Tan}^{-1}} 
\newcommand{\Area}{\text{Area}}
\newcommand{\vol}{\text{Vol}}
\newcommand{\maru}[1]{\raise0.2ex\hbox{\textcircled{\tiny{#1}}}}
\newcommand{\sgn}{{\rm sgn}}
%\newcommand{\rank}{{\rm rank}}



   %当然のようにやる.
\allowdisplaybreaks[4]
   %もちろん.
%\title{第1回. 多変数の連続写像 (岩井雅崇, 2020/10/06)}
%\author{岩井雅崇}
%\date{2020/10/06}
%ここまで今回の記事関係ない
\usepackage{tcolorbox}
\tcbuselibrary{breakable, skins, theorems}

\theoremstyle{definition}
\newtheorem{thm}{定理}
\newtheorem{lem}[thm]{補題}
\newtheorem{prop}[thm]{命題}
\newtheorem{cor}[thm]{系}
\newtheorem{claim}[thm]{主張}
\newtheorem{dfn}[thm]{定義}
\newtheorem{rem}[thm]{注意}
\newtheorem{exa}[thm]{例}
\newtheorem{conj}[thm]{予想}
\newtheorem{prob}[thm]{問題}
\newtheorem{rema}[thm]{補足}

\DeclareMathOperator{\Ric}{Ric}
\DeclareMathOperator{\Vol}{Vol}
 \newcommand{\pdrv}[2]{\frac{\partial #1}{\partial #2}}
 \newcommand{\drv}[2]{\frac{d #1}{d#2}}
  \newcommand{\ppdrv}[3]{\frac{\partial #1}{\partial #2 \partial #3}}


%ここから本文.
\begin{document}
%\maketitle

\newpage
\begin{center}
{\Large 2023年度春夏学期 大阪大学 理学部数学科 複素関数論続論演義
} \\
 火曜4限(15:10-16:40) 理学部D407
\end{center}
\begin{flushright}
 岩井雅崇(いわいまさたか) 2023/04/11 \\
\end{flushright}

\hspace{-18pt}{\Large 基本的事項}
\begin{itemize}
  \setlength{\parskip}{0cm} % 段落間
  \setlength{\itemsep}{0cm} % 項目間
\item この授業は対面授業です. 火曜4限(15:10-16:40)に理学部D407にて演習の授業を行います.
\item 授業ホームページ(\url{https://masataka123.github.io/2023_summer_complex/})にて授業の問題等をアップロードしていきます. 
下に授業ホームページのQRコードを貼っておきます. 
\begin{figure}[htbp]
\begin{center}
 \includegraphics[height=25mm, width=25mm]{complex.png}
 \caption*{授業のQRコード}
\end{center}
\end{figure}
\end{itemize}

\hspace{-18pt}{\Large 成績に関して}

次の1と2を満たしているものに単位を与えます.
\begin{enumerate}
  \setlength{\parskip}{0cm} % 段落間
  \setlength{\itemsep}{0cm} % 項目間
\item 複素関数論続論の講義の単位が可以上である.
\item 最終授業(2023年7月18日の予定)までに演習の授業で少なくとも1回以上発表している.
\end{enumerate}
%最終授業は2023年7月18日の予定です. %ただし2の条件を達成できないものには別途救済レポートを課して2を達成したものとすることがある.

演習の成績は"講義の成績"+"発表の成績"×(点数補正係数)でつける予定です. 発表の成績は「解いた問題の難易度」と「発表の仕方・解答の方法」などから定まります. 点数補正係数は実数かつ全員の成績から定まる係数です.

\vspace{11pt}

\hspace{-18pt}{\Large 解答の仕方について}
\begin{itemize}
  \setlength{\parskip}{0cm} % 段落間
  \setlength{\itemsep}{0cm} % 項目間
  \item 問題の解答を黒板に書いて発表してください. 正答だった場合その問題はそれ以降解答できなくなります. 不正解だった場合他の人に解答権が移ります. 
  \item 複数人が解答したい問題があるときは平和的な手段で解答者を決めます. 
  %\item 英語問題を答える際には英語を和訳してください. なお解答は日本語で行っても良い.
  \item 演習問題は適当に出しているので, 全部解く必要はないです.
 \end{itemize}
演習問題の難易度は一定ではないです. 問題番号の上に $\bullet$や $*$などの記号が書いてありますがこれは次を意味します.
\begin{itemize}
  \setlength{\parskip}{0cm} % 段落間
  \setlength{\itemsep}{0cm} % 項目間
\item $\bullet$がついてる問題は解けないといけない問題です. ある程度授業を理解している人は他の人に解答を譲ってください.
\item $*$問題や$**$問題は難しそうな問題です. ちょっと難しい問題から激ムズの問題まであります.
\item 何もついてない問題は普通の問題です. ちょっと考えれば解ける範疇に収まっている(はずです).
\end{itemize}


\vspace{11pt}
\hspace{-18pt}{\Large その他}
\begin{itemize}
  \setlength{\parskip}{0cm} % 段落間
  \setlength{\itemsep}{0cm} % 項目間
    \item 授業内容をあまり把握していないので, 演習問題と授業内容が噛み合ってない可能性があります.
    \item \underline{問題に間違いがある場合, すぐに教官(岩井)に言ってください.} 急いで仕上げたので間違いは多くあると思います. 
  \item 休講あるいは補講をすることがあるので, こまめにホームページとCLEは確認してください.
    \item オフィスアワーを月曜15:00-17:00に設けています. この時間に私の研究室に来ても構いません(ただし来る場合は前もって連絡してくれると助かります.)
    %\item $\pi$-base \url{https://topology.jdabbs.com}も活用してください. 
 \end{itemize}
 \begin{comment}


 \vspace{11pt}
\hspace{-18pt}{\large キーワードと重要度}

以下は個人的に思った重要度です. 
\begin{enumerate}
\item[重要度$\star\star\star$] 正則関数(コーシー・リーマン方程式)・コーシーの積分定理・リウヴィユの定理・ベキ級数・有理型関数(極と零点)・一致の定理
\item[重要度$\star\star$] ローラン展開・楕円関数($\wp$関数)・最大値原理
\item[重要度$\star$] コーシーの積分公式\footnote{コーシーの積分定理から簡単に出るから}・ワイエルシュトラスの二重級数定理・留数定理\footnote{コーシーの積分定理から簡単に出るから}
・偏角の原理・開写像原理・シュワルツの補題
\item[重要度] リーマンの写像定理\footnote{主張は面白いが定理の証明を覚えている人は限られるため}・フルビッツの定理・モンテルの定理
\end{enumerate}
 \end{comment}

 \end{document}
