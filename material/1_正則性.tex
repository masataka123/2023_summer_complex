\documentclass[dvipdfmx,a4paper,11pt]{article}
\usepackage[utf8]{inputenc}
%\usepackage[dvipdfmx]{hyperref} %リンクを有効にする
\usepackage{url} %同上
\usepackage{amsmath,amssymb} %もちろん
\usepackage{amsfonts,amsthm,mathtools} %もちろん
\usepackage{braket,physics} %あると便利なやつ
\usepackage{bm} %ラプラシアンで使った
\usepackage[top=30truemm,bottom=30truemm,left=25truemm,right=25truemm]{geometry} %余白設定
\usepackage{latexsym} %ごくたまに必要になる
\renewcommand{\kanjifamilydefault}{\gtdefault}
\usepackage{otf} %宗教上の理由でmin10が嫌いなので


\usepackage[all]{xy}
\usepackage{amsthm,amsmath,amssymb,comment}
\usepackage{amsmath}    % \UTF{00E6}\UTF{0095}°\UTF{00E5}\UTF{00AD}\UTF{00A6}\UTF{00E7}\UTF{0094}¨
\usepackage{amssymb}  
\usepackage{color}
\usepackage{amscd}
\usepackage{amsthm}  
\usepackage{wrapfig}
\usepackage{comment}	
\usepackage{graphicx}
\usepackage{setspace}
\usepackage{pxrubrica}
\usepackage{enumitem}
\usepackage{mathrsfs} 

\setstretch{1.2}


\newcommand{\R}{\mathbb{R}}
\newcommand{\Z}{\mathbb{Z}}
\newcommand{\Q}{\mathbb{Q}} 
\newcommand{\N}{\mathbb{N}}
\newcommand{\C}{\mathbb{C}} 
\newcommand{\D}{\mathbb{D}} 
%\newcommand{\H}{\mathbb{H}} 
\newcommand{\Sin}{\text{Sin}^{-1}} 
\newcommand{\Cos}{\text{Cos}^{-1}} 
\newcommand{\Tan}{\text{Tan}^{-1}} 
\newcommand{\invsin}{\text{Sin}^{-1}} 
\newcommand{\invcos}{\text{Cos}^{-1}} 
\newcommand{\invtan}{\text{Tan}^{-1}} 
\newcommand{\Area}{\text{Area}}
\newcommand{\vol}{\text{Vol}}
\newcommand{\maru}[1]{\raise0.2ex\hbox{\textcircled{\tiny{#1}}}}
\newcommand{\sgn}{{\rm sgn}}
%\newcommand{\rank}{{\rm rank}}



   %当然のようにやる.
\allowdisplaybreaks[4]
   %もちろん.
%\title{第1回. 多変数の連続写像 (岩井雅崇, 2020/10/06)}
%\author{岩井雅崇}
%\date{2020/10/06}
%ここまで今回の記事関係ない
\usepackage{tcolorbox}
\tcbuselibrary{breakable, skins, theorems}

\theoremstyle{definition}
\newtheorem{thm}{定理}
\newtheorem{lem}[thm]{補題}
\newtheorem{prop}[thm]{命題}
\newtheorem{cor}[thm]{系}
\newtheorem{claim}[thm]{主張}
\newtheorem{dfn}[thm]{定義}
\newtheorem{rem}[thm]{注意}
\newtheorem{exa}[thm]{例}
\newtheorem{conj}[thm]{予想}
\newtheorem{prob}[thm]{問題}
\newtheorem{rema}[thm]{補足}

\DeclareMathOperator{\Ric}{Ric}
\DeclareMathOperator{\Vol}{Vol}
 \newcommand{\pdrv}[2]{\frac{\partial #1}{\partial #2}}
 \newcommand{\drv}[2]{\frac{d #1}{d#2}}
  \newcommand{\ppdrv}[3]{\frac{\partial #1}{\partial #2 \partial #3}}


%ここから本文.
\begin{document}
%\maketitle

\begin{center}
{\Large 1 正則性}
\end{center}

\begin{flushright}
 岩井雅崇 2023/04/11
\end{flushright}
以下断りがなければ, $\Omega$は$\C$の領域(連結開集合)とする.

\begin{enumerate}[label=\textbf{問}1.\arabic*]

\item$^{\bullet}$次の問いに答えよ.
\begin{enumerate}
    \setlength{\parskip}{0cm} 
  \setlength{\itemsep}{0cm} 
  \item 「複素数$a_n$からなる級数$\sum_{n=1}^{\infty} a_n$が絶対収束する」ことの定義を述べよ.
  \item 複素数$a_n$からなる級数$\sum_{n=1}^{\infty}  a_n$が絶対収束するとする. このとき全単射$f : \N \rightarrow \N$について, $\sum_{n=1}^{\infty}  a_{f(n)} $も絶対収束して極限値は$\sum_{n=1}^{\infty}  a_n$であることを示せ.
\end{enumerate}

\item$^{\bullet}$次の問いに答えよ.
\begin{enumerate}
\setlength{\parskip}{0cm} 
  \setlength{\itemsep}{0cm} 
  \item 「$\Omega$上の関数列$f_{n}$が$\Omega$上の関数$f$に一様収束する」ことの定義を述べよ. 
  \item $\Omega$上の関数列$f_{n}$と$\Omega$上の関数$f$であって, 任意の$z \in \Omega$について$\lim_{n \rightarrow \infty} f_n(z) =f(z)$であるが, $f_{n}$が$f$に一様収束しない例をあげよ. 
  \item $\Omega$上の関数列$f_{n}$が$\Omega$上の関数$f$に一様収束すると仮定する. 任意の$n \in \N$について$f_{n}$が$\Omega$上で連続ならば, $f$も連続であることを示せ.
      \end{enumerate}


\item $^{\bullet}$(コーシー・リーマン方程式)$f(z)$を$\Omega$上の複素数値$C^{\infty}$級関数とする.
$z = x + iy \in \C$とし$f(z)$を$(x,y)$の関数$f(x,y)$と考え, 
$
f(x,y) = u(x,y) + i v(x,y)
$
とおく. ($u,v$は実数値$C^{\infty}$級関数とする.)
次は同値であることを示せ.
%\footnote{必要ならばコーシー・リーマン方程式を用いて良い.}
\begin{enumerate}
    \setlength{\parskip}{0cm} 
  \setlength{\itemsep}{0cm} 
  \item $f(z)$は$\Omega$上で正則である.
  \item $\pdrv{u}{x} = \pdrv{v}{y}$ かつ $\pdrv{v}{x} = - \pdrv{u}{y}$が成り立つ.
\end{enumerate}


  
 \item $^{\bullet}$次の関数は正則関数であるか判定せよ.
 \begin{enumerate}
\setlength{\parskip}{0cm} 
  \setlength{\itemsep}{0cm} 
  \item $f(z) = \bar{z}$
   \item $f(z) = |z|^2$
    \item $f(z) = \frac{1}{z}$ (ただし定義域は$\C \setminus \{ 0\}$とする.)
 \end{enumerate}
 
 \item $f(z)$を$\C$上の正則関数とする. このとき$g(z) = \overline{f(\bar{z})}$もまた$\C$上の正則関数であることを示せ. 

\item $f(z)$を$\Omega$上の正則関数とする. 任意の$z \in \Omega$について$f'(z)=0$となるならば$f$は定数関数であることを示せ.



\item \label{delbar}$z = x + iy \in \C$として複素偏微分を次で定義する.
$$
\pdrv{}{z} := \frac{1}{2} \left(\pdrv{}{x}  - i \pdrv{}{y} \right), \quad
\pdrv{}{\bar{z}} := \frac{1}{2} \left(\pdrv{}{x}  + i \pdrv{}{y}\right)
$$

$f(z)$を領域$\Omega  \subset \C$上の$C^{\infty}$級関数とするとき, 次は同値であることを示せ.
%\footnote{必要ならばコーシー・リーマン方程式を用いて良い.}
\begin{enumerate}
    \setlength{\parskip}{0cm} 
  \setlength{\itemsep}{0cm} 
  \item $f(z)$は$\Omega$上で正則である.
  \item $\Omega$上で$\pdrv{f}{\bar{z}}  \equiv 0$.
\end{enumerate}

\item 引き続き\ref{delbar}の通りの記号を用いる. 次の問いに答えよ. 
 \begin{enumerate}
\setlength{\parskip}{0cm} 
  \setlength{\itemsep}{0cm} 
\item  $f(z)$を$\Omega$上の正則関数とするとき, $f'(z)=\pdrv{f}{z}$であることを示せ.
\item $f(z)$を$\Omega$上の$C^{\infty}$級関数とするとき, 次を示せ.
$$\pdrv{}{z}\pdrv{}{\bar{z}} f(z) = \frac{1}{4}\left(\pdrv{}{x^2} + \pdrv{}{y^2} \right) f(z)$$
 \end{enumerate}
 

 \end{enumerate}

 
 \vspace{11pt}\begin{wrapfigure}{r}[0pt]{0.2\textwidth}  \centering\includegraphics[height=25mm, width=25mm]{complex.png}\end{wrapfigure}

演習の問題は授業ページ(\url{https://masataka123.github.io/2023_summer_complex/})にもあります. 右下のQRコードからを読み込んでも構いません.


  
  
 \end{document}
